\documentclass{article}



\usepackage{bibentry}
\usepackage{graphicx}
\usepackage{natbib}
\usepackage{amsmath}
\usepackage{url}
\usepackage{Zelig}
\usepackage{Sweave}
%\VignetteIndexEntry{Generalized Additive Model for Dichotomous Dependent Variables}
%\VignetteDepends{Zelig, stats}
%\VignetteKeyWords{model, logit.gam, gam}
%\VignettePackage{Zelig}
\begin{document}
\nobibliography*

\section{{\tt logit.gam}: Generalized Additive Model for Dichotomous Dependent Variables}\label{gam.logit}

This function runs a nonparametric Generalized Additive Model (GAM) for dichotomous dependent variables. 

\subsubsection{Syntax}
\begin{verbatim}
> z.out <- zelig(y ~ x1 + s(x2), model = "logit.gam", data = mydata) 
> x.out <- setx(z.out)
> s.out <- sim(z.out, x = x.out)
\end{verbatim}
Where {\tt s()} indicates a variable to be estimated via nonparametric smooth. All variables for which  {\tt s()} is not specified, are estimated via standard parametric methods. 

\subsubsection{Additional Inputs}
In addition to the standard inputs, {\tt zelig()} takes the following additional options for GAM models.
\begin{itemize}
\item {\tt method}: Controls the fitting method to be used. Fitting methods are selected via a list environment within {\tt method=gam.method()}. See {\tt gam.method()} for details. 
\item {\tt scale}:  Generalized Cross Validation (GCV) is used if {\tt scale = 0} (see the ``Model'' section for details) except for Logit models where a Un-Biased Risk Estimator (UBRE) (also see the ``Model'' section for details) is used with a scale parameter assumed to be 1. If {\tt scale} is greater than 1, it is assumed to be the scale parameter/variance and UBRE is used. If {\tt scale} is negative GCV is used.
\item {\tt knots}: An optional list of knot values to be used for the construction of basis functions. 
\item {\tt H}: A user supplied fixed quadratic penalty on the parameters of the GAM can be supplied with this as its coefficient matrix. For example, ridge penalties can be added to the parameters of the GAM to aid in identification on the scale of the linear predictor.
\item {\tt sp}: A vector of smoothing parameters for each term.
\item {\tt \ldots}: additional options passed to the {\tt logit.gam} model. See the {\tt mgcv} library for details. 
\end{itemize}


\subsubsection{Examples}
\begin{enumerate}
\item Basic Example

Create some count data:

\begin{Schunk}
\begin{Sinput}
> set.seed(0);  n <- 400; sig <- 2; 
> x0 <- runif(n, 0, 1);  x1 <- runif(n, 0, 1)
> x2 <- runif(n, 0, 1);  x3 <- runif(n, 0, 1)
> g <- (f-5)/3
> g <- binomial()$linkinv(g)
> y <- rbinom(g,1,g)
> my.data <- as.data.frame(cbind(y, x0, x1, x2, x3))
\end{Sinput}
\end{Schunk}

Estimate the model, summarize the results, and plot nonlinearities:

\begin{Schunk}
\begin{Sinput}
> z.out <- zelig(y ~ s(x0) + s(x1) + s(x2) + s(x3), model = "logit.gam", 
+     data = my.data)
> summary(z.out)
> plot(z.out, pages = 1, residuals = TRUE)
\end{Sinput}
\end{Schunk}
Note that the {\tt plot()} function can be used after model estimation and before simulation to view the nonlinear relationships in the independent variables: 

\begin{figure}[here]
\centering
